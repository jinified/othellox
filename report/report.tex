\documentclass[12pt]{article}
\usepackage{float}
\usepackage[margin=0.8in]{geometry}
\usepackage[utf8]{inputenc}
\usepackage[fleqn]{amsmath}
\usepackage{times}
\usepackage{hyperref}
\usepackage{multirow}
\usepackage{graphicx}
\usepackage{mathptmx}
\usepackage[font=scriptsize,labelfont=bf]{caption}
\graphicspath{{./img/}}

\title{CS3211 Project 2: OthelloX}
\author{Tan Soon Jin, A0112213E \\ a0112213@u.nus.edu}
\date{\today}

\begin{document}
\maketitle

%%%%%%%%%%%%%%%%%%%%%%%%%%%%%%%%%%%%%%%%%%%%%%%%%%%%%%%%%%%%%%%%%%%%%%%%%%%%%%%%%%%%%%%%%%%%%%%%%%%%

\section{Research}


%%%%%%%%%%%%%%%%%%%%%%%%%%%%%%%%%%%%%%%%%%%%%%%%%%%%%%%%%%%%%%%%%%%%%%%%%%%%%%%%%%%%%%%%%%%%%%%%%%%%

\section{Introduction}

\subsection{Problem Description}

Generating best moves given a board positions for the game of Othello / Reversi.

\subsection{Terminologies}

\begin{enumerate}
  \item Depth: move taken by a player
\end{enumerate}

\subsection{Minimax algorithm}

\subsection{Evaluation functions}

\begin{enumerate}
  \item \href{https://skatgame.net/mburo/ps/evalfunc.pdf}{An Evaluation Function for
  Othello Based on Statistics}
  \item
    \href{https://en.wikipedia.org/wiki/Computer_Othello#Evaluation_techniques}{Evaluation
    Techniques Wikipedia}
  \item
    \href{http://www.csse.uwa.edu.au/cig08/Proceedings/papers/8010.pdf}{Temporal
    Difference (TD) based evaluation function for Othello}
  \item \href{https://skatgame.net/mburo/ps/improve.pdf}{Multi-ProbCut and New
      Evaluation Function}
\end{enumerate}

\begin{enumerate}
  \item Complexity
  \item Limitations
\end{enumerate}
  

%%%%%%%%%%%%%%%%%%%%%%%%%%%%%%%%%%%%%%%%%%%%%%%%%%%%%%%%%%%%%%%%%%%%%%%%%%%%%%%%%%%%%%%%%%%%%%%%%%%%

\section{Data Structures \& Algorithms}

\subsection{Storage Techniques}

\subsection{Search algorithm}

\begin{enumerate}
  \item \href{http://www.aaai.org/Papers/AAAI/1994/AAAI94-210.pdf}{Best-First
      Minimax Search}
  \item
    \href{http://digitalarchive.maastrichtuniversity.nl/fedora/get/guid:36b5cf0a-cf06-4602-afdb-1af04d65c23b/ASSET1}{Searching
    for Solutions in Games and Intelligence}
\end{enumerate}

\subsection{Techniques to improve serialized components}

\subsection{Techniques to improve parallelization}

\begin{enumerate}
  \item \href{https://chessprogramming.wikispaces.com/Shared+Hash+Table}{Shared Hash Table}
  \item \href{https://chessprogramming.wikispaces.com/ABDADA}{Distributed Alpha-Beta Search with Eldest Son Right}
  \item \href{https://chessprogramming.wikispaces.com/Lazy+SMP}{Lazy SMP}
  \item \href{http://www.netlib.org/utk/lsi/pcwLSI/text/node350.html}{Parallel Alpha-Beta}
  \item
    \href{https://chessprogramming.wikispaces.com/Principal+Variation+Search}{Principal
    Variation Search}
  \item
    \href{https://chessprogramming.wikispaces.com/Dynamic+Tree+Splitting}{Dynamic
    Tree Splitting (DTS)}
  \item
    \href{https://dke.maastrichtuniversity.nl/m.winands/documents/multithreadedMCTS2.pdf}{Parallel
    Mont-carlo tree search}
  \item \href{http://www.lamsade.dauphine.fr/~cazenave/papers/rootparallelggp.pdf}{Parallel General Game Player}
\end{enumerate}

%%%%%%%%%%%%%%%%%%%%%%%%%%%%%%%%%%%%%%%%%%%%%%%%%%%%%%%%%%%%%%%%%%%%%%%%%%%%%%%%%%%%%%%%%%%%%%%%%%%%

\section{Tabulation of data}

\subsection{Testing scalability of the algorithms}

Scaling is measured using change in nodes per second (NPS). Speedup is measured
using change in time to depth.

\begin{enumerate}
  \item Depth of the search {2 - 8}
  \item Size of board {6x6, 8x10, ..., 26x26}
\end{enumerate}

%%%%%%%%%%%%%%%%%%%%%%%%%%%%%%%%%%%%%%%%%%%%%%%%%%%%%%%%%%%%%%%%%%%%%%%%%%%%%%%%%%%%%%%%%%%%%%%%%%%%

\section{Discussion}

%%%%%%%%%%%%%%%%%%%%%%%%%%%%%%%%%%%%%%%%%%%%%%%%%%%%%%%%%%%%%%%%%%%%%%%%%%%%%%%%%%%%%%%%%%%%%%%%%%%%

\bibliographystyle{acm}
\bibliography{ref}
\end{document}